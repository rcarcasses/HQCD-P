% Define document class
\documentclass{book}

% Call the needed packages
\usepackage{listings}
\usepackage{color}

% Setting the code language to appear
\definecolor{dkgreen}{rgb}{0,0.6,0}
\definecolor{gray}{rgb}{0.5,0.5,0.5}
\definecolor{mauve}{rgb}{0.58,0,0.82}

\lstset{frame=tb,
  language=R,
  aboveskip=3mm,
  belowskip=3mm,
  showstringspaces=false,
  columns=flexible,
  basicstyle={\small\ttfamily},
  numbers=none,
  numberstyle=\tiny\color{gray},
  keywordstyle=\color{blue},
  commentstyle=\color{dkgreen},
  stringstyle=\color{mauve},
  breaklines=true,
  breakatwhitespace=true,
  tabsize=3
}

\begin{document}
\title{Holographic QCD Pomeron}
\author{Artur Amorim, Robert Carcasses Quevedo}
\maketitle
\tableofcontents
%%%%%%%%%%%%%%%%%%%%%%%%
\chapter{Introduction}
%%%%%%%%%%%%%%%%%%%%%%%%

This is a manual that helps the user to use our tool. The objective is the user to choose his/her favourite model and comparing it with the available scattering data.

%%%%%%%%%%%%%%%%%%%%%%%%
\chapter{HQCDP Class}
%%%%%%%%%%%%%%%%%%%%%%%%

In order to fit the holographic model of QCD we need first to create a HQCDP object. Here we describe the functions of this class and provide examples of how to use them. Below we present the different functions of this class as well a description of how to use them.

%%%%%%%%%%%%%%%%%%%%%%%
\section{HQCDP}
%%%%%%%%%%%%%%%%%%%%%%%

This function starts a HQCDP object. It allows to define a model with many kernels that can be tested against the data of experimental observables available. It's definition is
\begin{lstlisting}
HQCDP <- function (alpha = 0, fixed = list (),
                  rsslog = FALSE, rootRejectionWeight = 1,
                  rootRejectionCutoff = 0.02, H = NULL,
                  hparsInitDefault = NULL)    
\end{lstlisting}
H is the function we will use to describe the couplings $k_J$ between the bulk fields and the fields of the graviton Regge trajectory.\ hparInitDefault is the parameters that define the function H.
To initialize a HQCDP object you just need to type
\begin{lstlisting}
    p <- HQCD ()
\end{lstlisting}

%%%%%%%%%%%%%%%%%%%%%%%
\section{addKernel}
%%%%%%%%%%%%%%%%%%%%%%%
This function allow to add a Kernel to the holographic pomeron. This kernel is determined by its potential, number of Regeons, a comment giving more details, its name and parameters necessary to define the potential.
After defining a HQCP object you can act in it with this function as
\begin{lstlisting}
    p <- addKernel (p, potential, numReg, comment, kernelName, optimPars)
\end{lstlisting}

%%%%%%%%%%%%%%%%%%%%%%%
\section{addProcessObservable}
%%%%%%%%%%%%%%%%%%%%%%%
In order to fit our model we need data. This function allows us to add which processes we want to use data to fit the holographic pomeron. If we define a ppDSigma object named pp we can add it as
\begin{lstlisting}
    p <- addProcessObservable (p, pp)
\end{lstlisting}

\end{document}