% Define document class
\documentclass{book}

% Call the needed packages to represent code
\usepackage{listings}
\usepackage{color}

% Call the needed packages to draw the decision tree
\usepackage{tikz,forest}
\usetikzlibrary{arrows.meta}
% Set the forest. Copied from stackexchange
\forestset{
    .style={
        for tree={
            base=bottom,
            child anchor=north,
            align=center,
            s sep+=1cm,
    straight edge/.style={
        edge path={\noexpand\path[\forestoption{edge},thick,-{Latex}] 
        (!u.parent anchor) -- (.child anchor);}
    },
    if n children={0}
        {tier=word, draw, thick, rectangle}
        {draw, diamond, thick, aspect=2},
    if n=1{%
        edge path={\noexpand\path[\forestoption{edge},thick,-{Latex}] 
        (!u.parent anchor) -| (.child anchor) node[pos=.2, above] {Y};}
        }{
        edge path={\noexpand\path[\forestoption{edge},thick,-{Latex}] 
        (!u.parent anchor) -| (.child anchor) node[pos=.2, above] {N};}
        }
        }
    }
}

% Setting the code language to appear
\definecolor{dkgreen}{rgb}{0,0.6,0}
\definecolor{gray}{rgb}{0.5,0.5,0.5}
\definecolor{mauve}{rgb}{0.58,0,0.82}

\lstset{frame=tb,
  language=R,
  aboveskip=3mm,
  belowskip=3mm,
  showstringspaces=false,
  columns=flexible,
  basicstyle={\small\ttfamily},
  numbers=none,
  numberstyle=\tiny\color{gray},
  keywordstyle=\color{blue},
  commentstyle=\color{dkgreen},
  stringstyle=\color{mauve},
  breaklines=true,
  breakatwhitespace=true,
  tabsize=3
}

\begin{document}

\title{Holographic QCD Pomeron}
\author{Artur Amorim, Robert Carcasses Quevedo}
\maketitle
\tableofcontents

%%%%%%%%%%%%%%%%%%%%%%%%
\chapter{Introduction}
%%%%%%%%%%%%%%%%%%%%%%%%

This is a manual that helps the user to use our tool. The objective is the user to choose his/her favourite model and comparing it with the available scattering data.

%%%%%%%%%%%%%%%%%%%%%%%%
\chapter{HQCDP Class}
%%%%%%%%%%%%%%%%%%%%%%%%

In order to fit the holographic model of QCD we need first to create a HQCDP object. Here we describe the functions of this class and provide examples of how to use them. Below we present the different functions of this class as well a description of how to use them.

%%%%%%%%%%%%%%%%%%%%%%%
\section{HQCDP}
%%%%%%%%%%%%%%%%%%%%%%%

This function starts a HQCDP object. It allows to define a model with many kernels that can be tested against the data of experimental observables available. It's definition is
\begin{lstlisting}
HQCDP <- function (alpha = 0, fixed = list (),
                  rsslog = FALSE, rootRejectionWeight = 1,
                  rootRejectionCutoff = 0.02, H = NULL,
                  hparsInitDefault = NULL)    
\end{lstlisting}
H is the function we will use to describe the couplings $k_J$ between the bulk fields and the fields of the graviton Regge trajectory.\ hparInitDefault is the parameters that define the function H.
To initialize a HQCDP object you just need to type
\begin{lstlisting}
    p <- HQCD ()
\end{lstlisting}

%%%%%%%%%%%%%%%%%%%%%%%
\section{addKernel}
%%%%%%%%%%%%%%%%%%%%%%%
This function allow to add a Kernel to the holographic pomeron. This kernel is determined by its potential, number of Regeons, a comment giving more details, its name and parameters necessary to define the potential.
After defining a HQCP object you can act in it with this function as
\begin{lstlisting}
    p <- addKernel (p, potential, numReg, comment, kernelName, optimPars)
\end{lstlisting}

%%%%%%%%%%%%%%%%%%%%%%%
\section{addProcessObservable}
%%%%%%%%%%%%%%%%%%%%%%%
In order to fit our model we need data. This function allows us to add which processes we want to use data to fit the holographic pomeron. If we define a ppDSigma object named pp we can add it as
\begin{lstlisting}
    p <- addProcessObservable (p, pp)
\end{lstlisting}
Remember to define first the object associated with the observable.

%%%%%%%%%%%%%%%%%%%%%%%
\section{rss}
%%%%%%%%%%%%%%%%%%%%%%%
For a given value of the parameters of our model it computes the sum of the $\chi^2$ of each individual observable. We try to minimize this quantity in order to find a optimal set of parameters. It takes as input a HQCDP object x, a set of parameters that characterizes the kernel of the pomeron as well the parameters that characterizes the couplings between the bulk fields and the fields of the graviton Regge trajectory. Simply call it by typing
\begin{lstlisting}
    rss (x, pars, zstar, hpars)
\end{lstlisting}

%%%%%%%%%%%%%%%%%%%%%%%
\section{fit}
%%%%%%%%%%%%%%%%%%%%%%%

Function that fits the model using the data of the processes that were added to the HQCDP object. To each of these processes it is associated a~$\chi^2$. The function we try to minimize is the sum of the $\chi^2$ of each process. It returns the minimum $\chi^2$ as well the values of the parameters of the model at that point. The standard method to find the minimum is Nelder-Mead.
\begin{lstlisting}
    fit (x, pars, zstar, hpars, method = `Nelder-Mead`)
\end{lstlisting}

%%%%%%%%%%%%%%%%%%%%%%%
\section{getNeededTVals}
%%%%%%%%%%%%%%%%%%%%%%%

For a given set of parameters, in order to predict observables, we need to compute the spectrum of the pomeron. The function getNeededTVals returns a list that contains all the necessary t values to compute de spectrum. This function is called after adding all the Process observable that we want to fit. Each of this process observable has a set of values of t that will be contained in the list returned by this function. For a HQCDP object x simply call
\begin{lstlisting}
    getNeededTVals (x)
\end{lstlisting}

%%%%%%%%%%%%%%%%%%%%%%%
\section{getSpectra}
%%%%%%%%%%%%%%%%%%%%%%%

After specifying the kernel of the Pomeron using the function addKernel we can compute the associated spectra. The input of this function is a HQCDP object, the set of parameters that characterizes the kernel and a list of t values for wich we want to compute the spectra. If the list of t values is not provided the function will call internally getNeededTVals. 

The function after called will return a list with the same number of elements as the number of different of different values of t. Each element has a t variable and a spectra variable that contains all the information relevant to the spectrum for a given value of t. The spectra variable then consists of a list whose elements correspond to the different kernels being used in the fit. Each of these kernels has a number of Reggeons that was specified when the kernel was added by the addKernel function. Each Reggeon is a list the the value of $j_n\left(t\right)$, $\frac{d j_n}{d t}$ and the corresponding wavefunction. An example of what to use this function is the following
\begin{lstlisting}
    x <- HQCDP ()
    x <- addKernel (p, potential = UJgTest,
                        numReg = 2, comment = `Leading twist gluon sector,
                        kernelName = `gluon,
                        optimPars = c (invls = 1/0.153, a = -4.35, b = 1.41, c = 0.626, d = -0.117))

    t <- seq (-10, 10, 0.1)
    spec <- getSpectra (x, t)
\end{lstlisting}
A scheme of how the output of getSpectra is organized is given below

\begin{forest}
    for tree={
      draw,
      minimum height=2cm,
      anchor=north,
      align=center,
      child anchor=north
    },
    [{getSpectra}, align=center, name=SS
      [{spectra}, name=PDC
        [K1, name=MS,
        [R1
        [$j_n$]
        [$\frac{d j_n}{dt}$]
        [wavefunction
        [x]
        [y]
        ]
        ]
        [R2]
        [R3]
        ]
        [K2
        [R1]
        [R2]
        ]
      ]
      [{t}]
    ]
    \node[anchor=west,align=left] 
      at ([xshift=-2cm]MS.west) {};
    \node[anchor=west,align=left] 
      at ([xshift=-2cm]MS.west|-PDC) {};
    \node[anchor=west,align=left] 
      at ([xshift=-2cm]MS.west|-SS) {};
    \end{forest}
\end{document}